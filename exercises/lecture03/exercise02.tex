\documentclass[11pt]{article}
\usepackage{times}
\usepackage[latin1]{inputenc}
\usepackage{ifthen}
%This is documentsubstyle DINA4 for DIN A4 pagesize.
  \topmargin 0mm
  \oddsidemargin 5mm
  \evensidemargin 5mm
  \textwidth 150mm
    % mods for 10 pt (\baselineskip=12pt)
    \textheight 634 pt  % = 222.5 mm
%% mods for 11 pt (\baselineskip=13.6pt)
%    \textheight 635.601 pt % = 223 mm  (!Rundungsfehler!)
%% mods for 12 pt  (\baselineskip=14.5pt)
%    \textheight 633.5 pt % = 222.8mm
 
\marginparwidth 0mm
\marginparsep 0mm
\marginparpush 0pt
\columnwidth\textwidth

\newcommand{\Veranstaltung}{Document~and~Content~Analysis~(SS~2009)}
\renewcommand{\thepage}{{\small\Veranstaltung}\hfill(\arabic{Blattnr})~\arabic{page}/\thelastpagenumber}
\makeatletter\def\thelastpagenumber{\textbf{??}}\AtEndDocument{\immediate\write\@auxout{\string\global\string\def\string\thelastpagenumber{\arabic{page}}}}\makeatother
\newcounter{Blattnr}
\newcommand{\AufgName}{Exercise}
\newtheorem{Aufg}{\AufgName}[Blattnr]
\newenvironment{Aufgabe}[2][]{%
 \ifthenelse{\equal{#1}{}}{\begin{Aufg}}{\begin{Aufg}[#1]}\normalfont\mbox{}#2\\
}{%
 \end{Aufg}%
}
\newenvironment{Aufgabenblatt}[3]{%
 \newpage%
 \setcounter{page}{1}%
 \setcounter{Blattnr}{#1}%
 \noindent
 Image~Understanding~and~Pattern~Recognition \hfill #2\\
 Prof.~Dr.~Thomas~M.~Breuel \hfill Dr.~Faisal~Shafait\\[2\baselineskip]
 \centerline{\textbf{\Large \Veranstaltung}}\\[\baselineskip]
 \centerline{\textbf{\Large\Vorsatz Exercise Sheet \theBlattnr}}\\[2\baselineskip]
 \AbgabeText{#3}\\
% \footnotesize
% \Rule
% Bearbeiten Sie zur Vertiefung der Vorlesung nach M�glichkeit stets alle Aufgaben!\\
% Mit Punkten ausgezeichnete Aufgaben k�nnen in den �bungen zur Korrektur abgegeben werden.\\
% Bitte kommentieren Sie ihre L�sungen dann stets ausf�hrlich!
% \\[-1.1ex]
 \Rule
 \par
 \normalsize
}{%
 \par
 \vfill
%%% \vspace{1ex}
 \noindent
 \Rule
 \texttt{\footnotesize http://courses.iupr.org}
 \par
}
\newcommand{\AbgabeText}[1]{\textbf{to be submitted by E-Mail to faisal.shafait@dfki.de by: #1}}
\newcommand{\Vorsatz}{}
\newcommand{\Vorlaeufig}{\fbox{Vorl�ufiges (!)} }
\newcommand{\Rule}[1][0pt]{\rule{\linewidth}{.5pt}\\[#1]}
\newcommand{\Punkte}[1]{\hfill(#1~Point\ifnum#1=1\else s\fi)}
\newcounter{Enumi}
\newcommand{\abccontinue}{\arabic{enumi}}
\newenvironment{abcenumerate}[1][0]{%
 \ifnum1=1#1\setcounter{Enumi}{\value{enumi}}\else\setcounter{Enumi}{#1}\fi%
 \begin{enumerate}\setcounter{enumi}{\value{Enumi}}\renewcommand{\labelenumi}{\alph{enumi})}%
}{%
 \end{enumerate}%
}
\newcommand{\abcenumerateUnskip}[1][-1.2\baselineskip]{\vspace{#1}}
\def\$#1${\mbox{$#1$}}

\usepackage{amsmath}
\begin{document}
\begin{Aufgabenblatt}{3}{18.06.2009}{29.06.2009}


Use Python 2.x for the following exercises.  Python 2.x has two
kinds of strings:
\begin{itemize}
\item regular strings (byte sequences), written as in {\tt "abc"}
\item unicode strings, written as in {\tt u"abc"}
\end{itemize}

Furthermore, the default encoding for strings in Pyton 2.x is set to
"ascii", which means that you will get an error if you try to print
anything that's not an ASCII character:
\begin{verbatim}
>>> print u"\u1000"
Traceback (most recent call last):
  File "<stdin>", line 1, in <module>
  UnicodeEncodeError: 'ascii' codec can't encode character u'\u1000' 
  in position 0: ordinal not in range(128)
\end{verbatim}
If you want to see interesting Unicode characters, you need to explicitly encode
your strings in your terminal's encoding.  The default terminal emulator
under Gnome uses a "utf-8" encoding.
If you just encode, the weird characters show up escaped:
\begin{verbatim}
>>> u"\u1000".encode('utf-8')
'\xe1\x80\x80'
>>> 
\end{verbatim}
If you actually want this 8-bit string sent to the console literally,
you need to print them explicitly:
\begin{verbatim}
>>> print u"\u1000".encode("utf-8")
...
>>> 
\end{verbatim}
However, you don't actually need to display any of these strings in order
to do the exercises (although you may find it useful).

Many Unicode fonts do not have glyphs for all Unicode characters; one
font that tries for complete coverage of the Basic Multilingual Plane
is the ``unifont'' font.  On Debian and Ubuntu, you can 
{\tt apt-get install ttf-unifont} (for other platforms, search for
Unifont on Google and follow the instructions).  To make a Gnome terminal
that uses Unifont, create a new profile and select ``unifont'' as the font.

You can put extended Unicode characters into strings using quoting:
\begin{itemize}
\item \verb|u"\u1234"| yields a character that is representable as a 16 bit
    codepoint.  This always takes four hex digits.
\item \verb|u"\U00010000| yields a character from an ``astral plane''.  This
    always takes 8 hex digits.
\end{itemize}

\begin{Aufgabe}[Python Unicode]{}
Write a pair of functions that convert a Python unicode string to
a list of integer codepoints and back.
\end{Aufgabe}

\begin{Aufgabe}[UTF-16]{}
Explain:
\begin{verbatim}
>>> s = u"\ud800\udc00"
>>> len(s)
2
>>> len(s.encode("utf-16").decode("utf-16"))
1
>>> 
\end{verbatim}
What happens if you use utf-8 instead of utf-16?
\par
Explain:
\begin{verbatim}
>>> len("x".encode("utf-16"))
4
>>> 
\end{verbatim}
\end{Aufgabe}

\begin{Aufgabe}[UTF-8]{}
Write a pair of functions, \verb|utf8_to_codepoints| and 
\verb|codepoints_to_utf8| for decoding and encoding UTF-8.  
\end{Aufgabe}

\begin{Aufgabe}[Unicode Leet Speak]{}
Write a function \verb|unileet(s)| that takes any string of
English letter, digits, spaces, hyphens, periods, and commas, 
and returns a readable Unicode string that uses no ASCII 
characters whatsoever.  The closer your resulting string looks
to the original, the better.
\end{Aufgabe}

\end{Aufgabenblatt}
\end{document}

\end{Aufgabe}

